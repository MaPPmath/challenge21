\documentclass[11pt,letterpaper]{article}

\usepackage[margin=1in]{geometry}

\usepackage{hyperref}

\linespread{1.2}

\setlength{\parindent}{2em}
\setlength{\parskip}{1em}

\title{
  Lamar University Mathematical Puzzlehunt \\
  Fall 2014 Project Proposal
}
\author{Steven Clontz}
\date{2014 May 04}

\newcommand{\signature}{
  \noindent\hspace{3.5in}Yours,

  \noindent\hspace{3.5in}Steven Clontz

  \vspace{-1em}
  \noindent\hspace{3.5in}steven.clontz@gmail.com

  \vspace{-1em}
  \noindent\hspace{3.5in}\url{http://stevenclontz.com}
}

\newcommand{\salutationTo}[1]{\noindent #1,}

\begin{document}

\maketitle


\salutationTo{Dr. PJ Couch}

Thank you for contacting me about organizing a mathematical puzzlehunt for
Lamar University. I'm excited for the opportunity to bring this unique kind
of mathematics outreach and recruitment event to your campus.

I've attached a project proposal outlining the scope of the event, our
respective responsibilities as we make preparations and run the competition, 
and the costs associated. Please let me know any changes that you would
like made to this proposal, and I'll be happy to make adjustments to customize
this event to fit the needs and goals of your department.

\signature



\newpage

\section{Outline of Event}

\subsection{Abstract}

The Fall 2014 \textit{Lamar University Mathematical Puzzlehunt} (title pending)
is a team-based
competition for 9th-12th grade students. Each team is composed of up to
eight students, with at least six students per team suggested. A teacher
chaperone is also required for each team, but the chaperone cannot assist
the team in the competition.

Teams will be presented with six puzzles testing their deductive reasoning
ability and introducing concepts from mathematical fields such as cryptography, 
game theory, or others. Solving puzzles will require the team to run out and
interact with Lamar University's campus to collect solution tokens. Tokens will 
be shown to the organizer at Staff Headquarters to receive credit, and the team 
which solves the most puzzles the fastest wins the competition.

\subsection{Schedule}

Date of Event: Saturday, 2014 October 18

\begin{itemize}
  \item 10:30am: Event setup
  \item 11:45pm: Registration
  \item 12:15pm: Orientation
  \item 12:30pm: Puzzlehunt begins
  \item 4:30pm: Deadline for turning in solutions
  \item 4:40pm: Wrap-up, Awards, Dismissal
  \item 5:00pm: Event tear-down
\end{itemize}



\newpage

\section{Responsibilities}

\subsection{Designer}

The competition designer, Steven Clontz, is responsible for the creation and
execution of the puzzlehunt. His responsibilities include:

\begin{itemize}
  \item Creation of a teaser puzzle in PDF format for use in promotion of
        the event.
  \item Designing the competition, including rules for the game, one
        opening run-around challenge, and six mathematical puzzles.
  \item Production of PDFs for all printed materials, including rule
        handbooks, puzzle descriptions, and puzzle solutions
  \item Acquisition and/or creation of any required props for the puzzles.
  \item Setting up the event prior to registration with the assistance
        of the host staff
  \item Presenting a short orientation for teams prior to the competition
  \item Running the competition, including
    \begin{itemize}
      \item Answering game-related questions from players and chaperones
      \item Tallying team scores
      \item Correcting any errata or ambiguity
    \end{itemize}
  \item Presenting a short wrap-up of the game following the competition
\end{itemize}

\subsection{Host}

The event host, Lamar University, is responsible for the marketing of the
event, as well as managing the necessary background logistics for the event.
The responsibilities of the event host's representatives include:

\begin{itemize}
  \item Marketing the event, including all contact with school
        representatives before the day of the event and any promotional
        materials or websites (aside from the promotional teaser puzzle).
  \item Recruiting teams to participate in the event
  \item Creation of a logo for use in materials related to the event.
  \item Printing, laminating, and binding materials from PDFs provided by the 
        designer.
  \item Purchasing notepads and writing utensils for use by the players
  \item Acquiring awards/plaques/trophies etc. for participating/winning teams.
  \item Playtesting and otherwise providing timely feedback to the designer on
        puzzles, game rules, and other topics related to the event
  \item Transportation for the event designer to/from IAH and Lamar University
        the weekend of the event, and transportation while on campus
  \item Providing locations for the event, including
    \begin{itemize}
      \item A large lecture hall or auditorium for Registration, Orientation,
            and Wrap-Up/Awards
      \item Any required permissions for outdoor space to be used in an opening 
            run-around challenge and storing solution tokens
      \item Classrooms for each participating team to be used during the
            competition as team headquarters
      \item A staff headquarters for the designer and staff to use during
            the competition
    \end{itemize}
  \item Providing staff on the day of the event, to help with
    \begin{itemize}
      \item Assisting the designer with event setup
      \item Running registration for the event and distributing materials
      \item Introducing Lamar and the designer at orientation
      \item Answering non-game related questions from students and chaperones
      \item Guiding teams around campus as needed
      \item Presenting awards after the competition
    \end{itemize}
  \item Providing any drinks or snacks for participants during the competition
  \item Surveying participants for assessment of event
\end{itemize}



\newpage

\section{Project Timeline}

\subsection{May/June 2014}

Project proposal is finalized and accepted. Invoice for down-payment is 
sent and paid. Designer begins writing puzzles and designing the game.

\subsection{July/August 2014}

Rough draft of puzzles and game rules is sent from designer to Lamar for
review. Feedback and revisions. 

\subsection{August/September 2014}

Final draft of puzzles and game rules is sent from designer to Lamar for 
review. Playtesting by both designer and Lamar, with feedback. Final
revisions and game is finalized by end of September.

\subsection{October 2014}

Event runs on October 18th. Invoice for payment balance is sent
and paid. Lamar and designer reflect together on the event
and review survey results from participants.



\newpage

\section{Compensation}

Lamar University will pay the game designer Steven Clontz a total of \$2000 
in exchange for contributing his time and expertise toward organizing
this unique mathematics competition. This money will also cover lodging and 
travel expenses for the organizer to/from the event, as well as other 
miscellaneous expenses occurred in the preparation of the event.

The designer will send an invoice for \$750 upon the acceptance of the
project proposal, and will begin work on the project upon receipt of this
down payment. The designer will invoice the remaining \$1250 after the
completion of the event.

\section{Contact} 

Contact between Lamar and the designer will primarily be through email, but
will also be available by phone as required.

\subsection{Contact information} 

\noindent Designer: \\
Steven Clontz \\
steven.clontz@gmail.com \\
(256) 508-3864

\noindent Lamar University Representative: \\
Dr. PJ Couch \\
pj.couch@lamar.edu \\
(409) 420-4060


\end{document}






