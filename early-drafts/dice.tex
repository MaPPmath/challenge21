\documentclass{article}
\usepackage[margin=1in]{geometry}

\usepackage{pifont}
\DeclareFontFamily{U}{dice3d}{}
\DeclareFontShape{U}{dice3d}{m}{n}{<-> s*[20] dice3d}{}
\usepackage{tikz}

\newcommand{\die}[2]{
\begin{scope}[shift={(#1)}]
\fill[color=white] (0,0.2) -- (0.2,0) -- (0.9,0) -- (0.95,0.1) -- (0.95,1)
  -- (0.03,3.65) -- (-0.9,3.7) -- (-0.9,2.85);
\node[anchor=south west] at (0,0) {\usefont{U}{dice3d}{m}{n}{#2}};
\end{scope}
}
\newcommand{\diegridcustom}[2]{
\begin{tikzpicture}[every node/.style={inner sep=0,outer sep=0},scale=1.34,y={(0.35355,0.35355,0)}]
\foreach \x in {0,1,...,#2} {
\draw[thick] (0,\x) -- (#2,\x);
\draw[thick] (\x,0) -- (\x,#2);
}
#1
\end{tikzpicture}
}
\newcommand{\diegrid}[1]{\diegridcustom{#1}{4}}

\begin{document}
\begin{center}
\diegrid{
    \die{0,2}{1d}
    \die{2,3}{3a}
}
\diegrid{
    \die{0,1}{6a}
    \die{3,3}{5a}
}

\diegrid{
    \die{0,1}{4b}
    \die{2,3}{5a}
}
\diegrid{
    \die{1,1}{2b}
    \die{3,3}{1b}
}

\diegrid{
    \die{0,2}{4c}
    \die{2,3}{2a}
}
\diegrid{
    \die{1,1}{2d}
    \die{2,2}{1b}
}

\diegrid{
    \die{0,1}{5a}
    \die{1,3}{2b}
}
\diegrid{
    \die{1,1}{4d}
    \die{3,2}{2d}
}

\diegrid{
    \die{1,1}{2c}
    \die{2,2}{4b}
}
\diegrid{
    \die{1,2}{3d}
    \die{3,3}{2c}
}

\diegrid{
    \die{1,1}{2a}
    \die{2,2}{2a}
}
\diegrid{
    \die{1,1}{4d}
    \die{3,3}{1b}
}
\end{center}

\newpage

Flavortext:

One of the Designer's hobbies
was dice games. Strangely, it's said that
they would spend hours simply rolling
a die across a grid, muttering
"all that matters is the pips
I can see". We believe that
one of their passwords is named after
a city location where the Designer
would play such dice games.

In the illustration below, the
upper die may be obtained by
rolling the lower die one space
up the grid,
then one space right.
Or it could be obtained by moving
left, down, right, up, up,  up,
right, down, left, up,
right, and down.
%But I'm not sure what
%that T is supposed to mean..


%The second illustration shows many
%other examples: any die shown may
%be rolled into any other position.
Many things are possible in The Array,
but maybe not every dice roll is...

\begin{center}
\diegrid{
    \die{1,1}{1a}
    \die{2,2}{4d}
}

%T

%\vspace{1em}

%\diegridcustom{
%    \die{1,0}{1d}
%    \die{3,4}{2c}
%    \die{5,0}{3a}
%    \die{6,2}{4c}
%    \die{0,6}{5a}
%    \die{4,7}{6d}
%}{8}
\end{center}
\end{document}
